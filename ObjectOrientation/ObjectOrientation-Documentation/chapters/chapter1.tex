\section{Introduzione}

Questa documentazione fornisce una guida esaustiva sull'implementazione e sull'uso del sistema di gestione della logistica delle spedizioni di merci di \textbf{UninaDelivery}. L'applicativo, progettato con un'interfaccia grafica intuitiva, è destinato agli operatori di un'azienda di delivery, consentendo loro di gestire le operazioni logistiche in modo efficiente, preciso e organizzato.

Il sistema offre le seguenti funzionalità principali:
\begin{itemize}[leftmargin=*,label={\textbullet},itemsep=0pt,topsep=0pt,partopsep=0pt]
  \item Visualizzare in tempo reale la lista degli ordini in attesa di evasione;
  \item Pianificare e creare spedizioni per tali ordini, ottimizzando le risorse e tenendo conto della disponibilità delle merci nei depositi;
  \item Generare e consultare report mensili sugli ordini evasi per migliorare supervisione, monitoraggio e pianificazione strategica.
\end{itemize}

\subsection{Assunzioni sul dominio}

Il sistema \textbf{UninaDelivery} è stato progettato per un'azienda che gestisce spedizioni di:
\begin{itemize}[leftmargin=*,label={\textbullet},itemsep=0pt,topsep=0pt,partopsep=0pt]
  \item \textbf{Merce propria}, prodotta internamente;
  \item \textbf{Merce di terzi}, gestita in custodia per conto di clienti esterni.
\end{itemize}
L'azienda opera a livello \textbf{nazionale}, coprendo l'intero territorio con una rete di depositi distribuiti.

Gli operatori possono gestire sia spedizioni dirette verso i clienti, sia spostamenti intermedi tra depositi per ottimizzare la disponibilità di merci nei punti strategici. Inoltre, si assume che:
\begin{itemize}[leftmargin=*,label={\textbullet},itemsep=0pt,topsep=0pt,partopsep=0pt]
  \item Le spedizioni dirette verso un cliente avvengano preferibilmente da depositi nella stessa città del destinatario.
  \item In caso di indisponibilità di prodotti nel deposito locale, il sistema supporti spedizioni intermedie da altri depositi o spostamenti interni, garantendo la completa evasione dell'ordine.
  \item Ogni ordine sia limitato a un solo prodotto, con quantità variabile. Gli ordini complessi (contenenti più prodotti) vengono suddivisi automaticamente in ordini separati, a condizione che siano destinati alla stessa area geografica (identificata dal CAP) o al medesimo deposito finale.
\end{itemize}

\subsection{Obiettivo della documentazione}

Questa documentazione mira a fornire una guida dettagliata e completa sulle funzionalità del sistema \textbf{UninaDelivery}, coprendo tutti gli aspetti fondamentali:
\begin{itemize}[leftmargin=*,label={\textbullet},itemsep=0pt,topsep=0pt,partopsep=0pt]
  \item Configurazione iniziale e requisiti di sistema;
  \item Utilizzo quotidiano per la gestione operativa delle spedizioni;
  \item Procedure di manutenzione e aggiornamento del software.
\end{itemize}

Il documento è pensato per diversi tipi di utenti:
\begin{itemize}[leftmargin=*,label={\textbullet},itemsep=0pt,topsep=0pt,partopsep=0pt]
  \item \textbf{Operatori logistici}, che utilizzeranno il sistema per la gestione pratica delle spedizioni;
  \item \textbf{Amministratori di sistema}, responsabili della configurazione e manutenzione tecnica;
  \item \textbf{Analisti e supervisori}, che sfrutteranno i report per monitorare e ottimizzare le attività logistiche.
\end{itemize}

Sia che siate nuovi al sistema o utenti esperti in cerca di informazioni dettagliate, questa guida è stata progettata per essere un riferimento intuitivo e affidabile.