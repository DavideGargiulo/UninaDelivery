\section{Introduzione}

\subsection{Scopo Del Documento}

Questo documento fornisce una panoramica dettagliata di UninaDelivery, un sistema per la gestione della logistica e delle spedizioni di merci. Il documento è stato concepito per offrire una guida completa agli utenti del sistema, delineando le sue funzionalità, architettura e il flusso di lavoro associato.

\subsection{Descrizione Del Sistema}

\normalsize{Il database di UninaDelivery è stato progettato per ottimizzare la gestione logistica delle spedizioni di merci basandosi sugli ordini dei clienti. Il sistema permette agli operatori di pianificare in modo efficiente le spedizioni, tenendo conto di variabili cruciali come la disponibilità della merce, il suo peso, e la disponibilità di mezzi di trasporto e corrieri. Questa soluzione tecnologica mira a migliorare l'efficienza operativa, ridurre i tempi di consegna e massimizzare la soddisfazione del cliente.}

\subsection{Utenti Intesi/Destinatari}

Questa documentazione è stata creata pensando in particolare ai seguenti destinatari:

\begin{itemize}[leftmargin=*,label={\textbullet},itemsep=-1pt,topsep=3pt,partopsep=0pt]
    \item \normalsize{\textbf{Operatori Logistici:}} che utilizzeranno il sistema quotidianamente per la gestione e pianificazione delle spedizioni;
    \item \textbf{Team di Supporto e Manutenzione:} responsabili della manutenzione e dell'aggiornamento del sistema;
    \item \textbf{Dirigenti Aziendali:} che necessitano di comprendere le capacità e i benefici del sistema per prendere decisioni informate a livello strategico;
    \item \textbf{Sviluppatori di Software:}  che potrebbero avere bisogno di interfacciarsi con UninaDelivery per integrazioni o personalizzazioni.
\end{itemize}