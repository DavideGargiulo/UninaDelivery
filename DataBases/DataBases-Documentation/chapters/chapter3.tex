\section{Schema Fisico}

\subsection{Creazione delle Tabelle}

\subsubsection{Account}

\begin{lstlisting}
    CREATE TABLE Account (
        Name VARCHAR(255),
        Surname VARCHAR(255),
        Email VARCHAR(255) PRIMARY KEY,
        Password VARCHAR(255),
        Birthdate DATE
    );
	ALTER TABLE Account
    ADD CONSTRAINT alfanumName CHECK (Name ~ '^[A-Za-z0-9]+$'),
    ADD CONSTRAINT alfanumSurname CHECK (Surname ~ '^[A-Za-z0-9]+$'),
    ADD CONSTRAINT alfanumEmail CHECK (Email ~ '^[A-Za-z0-9@.]+$');
\end{lstlisting}

\subsubsection{CourierVehicle}

\begin{lstlisting}
    CREATE TABLE CourierVehicle (
        LicensePlate VARCHAR(255) PRIMARY KEY,
        Model VARCHAR(255),
        MaxCapacity INT
    );
    ALTER TABLE CourierVehicle
    ADD CONSTRAINT capacityMaxVehicleCheck CHECK (MaxCapacity > 0);
\end{lstlisting}

\subsubsection{Driver}

\begin{lstlisting}
    CREATE TABLE Driver (
        DriverId INT PRIMARY KEY,
        BusinessEmail VARCHAR(255),
        Email VARCHAR(255),
        LicensePlate VARCHAR(255),
        FOREIGN KEY (Email) REFERENCES Account(Email),
        FOREIGN KEY (LicensePlate) REFERENCES CourierVehicle(LicensePlate)
    );
    ALTER TABLE Driver
    ADD CONSTRAINT alfanumDriverEmail CHECK (BusinessEmail ~ '^[A-Za-z0-9@.]+$');
\end{lstlisting}

\newpage

\subsubsection{Operator}

\begin{lstlisting}
    CREATE TABLE Operators (
        OperatorId INT PRIMARY KEY,
        BusinessEmail VARCHAR(255),
        Email VARCHAR(255),
        FOREIGN KEY (Email) REFERENCES Account(Email)
    );
    ALTER TABLE Operators
    ADD CONSTRAINT alfanumOperatorEmail CHECK (BusinessEmail ~ '^[A-Za-z0-9@.]+$');
\end{lstlisting}

\subsubsection{Warehouse}

\begin{lstlisting}
    CREATE TABLE Warehouse (
        WarehouseId INT PRIMARY KEY,
        Name VARCHAR(255),
        MaxCapacity INT,
        OccupiedSpace INT,
        VAT VARCHAR(255),
        ShipmentId INT,
        FOREIGN KEY (VAT) REFERENCES Company(VAT),
        FOREIGN KEY (ShipmentId) REFERENCES Shipment(ShipmentId)
    );
    ALTER TABLE Warehouse
    ADD CONSTRAINT capacityMaxCheck CHECK (MaxCapacity > 0),
    ADD CONSTRAINT occupiedSpaceCheck CHECK (OccupiedSpace >= 0);
\end{lstlisting}

\subsubsection{Shipment}

\begin{lstlisting}
    CREATE TABLE Shipment (
        ShipmentId INT PRIMARY KEY,
        ScheduledDate DATE,
        Price DECIMAL(10, 2),
        OperatorId INT,
        FOREIGN KEY (OperatorId) REFERENCES Operators(OperatorId)
    );
\end{lstlisting}

\subsubsection{Parcel}

\begin{lstlisting}
    CREATE TABLE Parcel (
        ParcelId INT PRIMARY KEY,
        Dimension VARCHAR(255),
        Status VARCHAR(255),
        Weight DECIMAL(10, 2),
        ShipmentId INT,
        FOREIGN KEY (ShipmentId) REFERENCES Shipment(ShipmentId)
    );
\end{lstlisting}

\newpage

\subsubsection{Company}

\begin{lstlisting}
    CREATE TABLE Company (
        VAT VARCHAR(255) PRIMARY KEY,
        Name VARCHAR(255)
    );
\end{lstlisting}

\subsubsection{Product}

\begin{lstlisting}
    CREATE TABLE Product (
        ProductId INT PRIMARY KEY,
        Name VARCHAR(255),
        Image VARCHAR(255),
        Description TEXT,
        Price DECIMAL(10, 2),
        CreatedAt DATE,
        Category VARCHAR(255),
        IsFragile BOOLEAN
    );
    ALTER TABLE Product
    ADD CONSTRAINT alfanumCategory CHECK (Category ~ '^[A-Za-z0-9]+$'),
    ADD CONSTRAINT alfanumProductName CHECK (Name ~ '^[A-Za-z0-9]+$'),
    ADD CONSTRAINT alfanumDescription CHECK (Name ~ '^[A-Za-z0-9]+$'),
    ADD CONSTRAINT productPriceCheck CHECK (Price > 0);
\end{lstlisting}

\subsubsection{Order}

\begin{lstlisting}
    CREATE TABLE Orders (
        OrderId INT PRIMARY KEY,
        Price DECIMAL(10, 2),
        CreatedAt DATE,
        Quantity INT,
        Email VARCHAR(255),
        ProductId INT,
        FOREIGN KEY (Email) REFERENCES Account(Email)
    );
\end{lstlisting}

\newpage

\subsubsection{SavedAddress}

\begin{lstlisting}
    CREATE TABLE SavedAddress (
        AddressId INT PRIMARY KEY,
        City VARCHAR(255),
        Province VARCHAR(255),
        ZipCode VARCHAR(255),
        Street VARCHAR(255),
        CivicNumber VARCHAR(255),
        State VARCHAR(255),
        Name VARCHAR(255),
        VAT VARCHAR(255),
        WarehouseId INT,
        Email VARCHAR(255),
        FOREIGN KEY (Name) REFERENCES Courier(Name),
        FOREIGN KEY (VAT) REFERENCES Company(VAT),
        FOREIGN KEY (WarehouseId) REFERENCES Warehouse(WarehouseId),
        FOREIGN KEY (Email) REFERENCES Account(Email)
    );
\end{lstlisting}

\subsubsection{Stores}

\begin{lstlisting}
    CREATE TABLE Stores (
        Quantity INT,
        WarehouseId INT,
        ProductId INT,
        FOREIGN KEY (WarehouseId) REFERENCES Warehouse(WarehouseId),
        FOREIGN KEY (ProductId) REFERENCES Product(ProductId)
    );
\end{lstlisting}

\subsubsection{Courier}

\begin{lstlisting}
    CREATE TABLE Courier (
        Name VARCHAR(255) PRIMARY KEY,
        LicensePlate VARCHAR(255),
        FOREIGN KEY (LicensePlate) REFERENCES CourierVehicle(LicensePlate)
    );
\end{lstlisting}

\subsubsection{TransportedBy}

\begin{lstlisting}
    CREATE TABLE TransportedBy (
        ShipmentId INT,
        Name VARCHAR(255),
        FOREIGN KEY (ShipmentId) REFERENCES Shipment(ShipmentId),
        FOREIGN KEY (Name) REFERENCES Courier(Name)
    );
\end{lstlisting}

\subsubsection{ShippedTo}

\begin{lstlisting}
    CREATE TABLE ShippedTo (
        ShipmentId INT,
        Email VARCHAR(255),
        FOREIGN KEY (ShipmentId) REFERENCES Shipment(ShipmentId),
        FOREIGN KEY (Email) REFERENCES Account(Email)
    );
\end{lstlisting}

\subsubsection{Includes}

\begin{lstlisting}
    CREATE TABLE Includes (
        OrderId INT,
        ProductId INT,
        FOREIGN KEY (OrderId) REFERENCES Orders(OrderId),
        FOREIGN KEY (ProductId) REFERENCES Product(ProductId)
    );
\end{lstlisting}

\subsection{Definizione del Trigger}

\subsubsection{checkBirthdateDriverOperator}

\begin{lstlisting}
    CREATE OR REPLACE FUNCTION checkAgeBeforeInsertOrUpdateDriver()
    RETURNS TRIGGER AS $$
    DECLARE
        age INT;
    BEGIN
        -- Calcola l'età basata sulla data di nascita
        SELECT EXTRACT(YEAR FROM age(CURRENT_DATE, Birthdate)) INTO age
        FROM Account WHERE Email = NEW.Email;

        -- Controlla se l'età è inferiore a 18
        IF age < 18 THEN
            RAISE EXCEPTION 'L'autista deve avere almeno 18 anni.';
        END IF;

        RETURN NEW;
    END;
    $$ LANGUAGE plpgsql;

    CREATE TRIGGER checkBirthdateDriver
    BEFORE INSERT ON Driver
    FOR EACH ROW
    EXECUTE FUNCTION checkAgeBeforeInsertOrUpdateDriver();
\end{lstlisting}

\newpage

\subsubsection{updateCapacityAfterUpdate}

\begin{lstlisting}
    CREATE OR REPLACE FUNCTION update_warehouse_capacity()
    RETURNS TRIGGER AS $$
    BEGIN
        UPDATE Warehouse
        SET OccupiedSpace = OccupiedSpace + NEW.Quantity - OLD.Quantity
        WHERE WarehouseId = NEW.WarehouseId;
        RETURN NEW;
    END;
    $$ LANGUAGE plpgsql;

    CREATE TRIGGER updateCapacityAfterUpdate
    AFTER UPDATE ON Stores
    FOR EACH ROW EXECUTE FUNCTION update_warehouse_capacity();
\end{lstlisting}

\subsection{Definizione della Funzione}

\begin{lstlisting}
    CREATE OR REPLACE FUNCTION process_new_order(
        p_order_id INT,
        p_product_id INT,
        p_email VARCHAR,
        p_quantity INT,
        p_price DECIMAL(10,2),
        p_created_at DATE
    )
    RETURNS VOID AS $$
    DECLARE
        available_quantity INT;
    BEGIN
        -- Query per ottenere la quantità disponibile del prodotto richiesto dal magazzino.
        SELECT Quantity INTO available_quantity FROM Stores WHERE ProductId = p_product_id;
        IF available_quantity < p_quantity THEN
            RAISE EXCEPTION 'Quantità insufficiente nel magazzino.';
        END IF;

        -- Inserimento del nuovo ordine nella tabella Orders.
        INSERT INTO Orders (OrderId, Price, CreatedAt, Quantity, Email, ProductId)
        VALUES (p_order_id, p_price, p_created_at, p_quantity, p_email, p_product_id);

        -- Aggiornamento della quantità del prodotto nel magazzino
        UPDATE Stores
        SET Quantity = Quantity - p_quantity
        WHERE ProductId = p_product_id;
    END;
    $$ LANGUAGE plpgsql;
\end{lstlisting}